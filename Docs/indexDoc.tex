\documentclass[a4paper]{article}

\newcommand{\cernvmgraphics}{CernVM-Graphics}
\newcommand{\json}{JSON}
\title{Index.json - The Configuration File For \cernvmgraphics{}}
\author{Ben Page}

\begin{document}
\maketitle

\section{Overview}
  index.json is a \json{} formatted file that describes what appears in
  \cernvmgraphics{} - think of it like HTML to any web browser. Like any
  \json{} file we require a root element.  The root element contains three 
  nodes called ``resources'', ``sprites" and ``objects''. ``resources'' is 
  an object which describes external ``resource'' files; ``sprites'' is an 
  object that describes the location of the images or 'sprites' that 
  appear in the program and ``objects'' is a list of json objects that 
  describe controls/widgets or 'objects' that appear in the 
  \cernvmgraphics{}.

\section{Resources}
  ``resources'' is formatted as a dictionary of ``key'' : ``value'' pairs 
  where the key describes the internal name of the resource, and the value 
  the filename. This allows us to externalise things such as images 
  (sprites) and metadata, as the resource provider may not care about how
  it is displayed. Resource files are simply a root node that contains
  a collection of nodes that are requested by the rest of the program. So
  if you have externalised your sprites then  your resources file may look
  something like this:

  \begin{verbatim}
  {
    "sprites":
    {
      "__main__" : 
      {
        "sprite1" : "file1.png",
        "sprite2" " "file2.png"
      }
    }
  }
  \end{verbatim}

  The exact formatting of how to do things like this is shown in the 
  following sections.

\section{Sprites}
  \subsection{The ``sprites'' node and sprite groups}
  ``sprites'' is an object that either describes the sprites to be used, or
  gives a resource node that describes the sprites to be used. The json
  description of sprites to be used is the same either inside or outside of
  this file. To give a resource node the ``sprites'' elements looks such:

  \begin{verbatim}
    "sprites" :
    {
      "external" : true,
      "resource" : resourceName,
      "node"     : nodeName
    }
  \end{verbatim}

  The boolean ``external'' is required to be true if the sprites are stored
  externally, in a resource. The internal name of the resource (as defined 
  in section 1.1) is provided as the value to the key ``resource''. The json
  node within this resource is provided as the value to the key ``node''.

  As noted before, internal or external sprite descriptions are the same.
  The first thing to note is that sprites are always given in ``groups".
  This is because certain ``objects" can take an entire group of sprites at
  one time (for example, a slideshow uses a group of sprites).

  The ``sprites'' element is a dictionary whose keys are the group names and
  values are a list of objects which describe the sprite settings. For
  example:

  \begin{verbatim}
    "sprites" :
    {
      "myGroup" : 
      [
        {"name": "sprite1", "file" : "sprite1.png"},
        {"name": "spritez", "file" : "spritez.png"}
      ]
    }
  \end{verbatim}

  This defines a group of sprites called ``myGroup''. Within this group we
  have two sprites - named ``sprite1'' and ``spritez'' which load files
  ``sprite1.png'' and ``spritez.png'' respectively, from the VM.

  To be able to refer to sprites without the group name we have reserved one
  group name - ``\_\_main\_\_''. Any sprites placed in this group will be
  accessible (within an object) by internal name only, no need to state the 
  group.

  \subsection{Sprite objects}
  Each sprite object can be of two different types - a full object, or just
  a string that describes the file location. That is a group defined as
  below is entirely valid:

  \begin{verbatim}
    "myGroup" : 
    [
      "sprite1.png",
      {"name": "spritez", "file" : "spritez.png"}
    ]
  \end{verbatim}

  It is, however, very difficult (and there is no standard way) to get at
  the ``nameless'' sprite via an object. However an object (for example a
  slideshow) will still be able to use it, if it is given the entire group
  to handle. This comes in handy if you simply want to set up a series of
  slides, and only care that they're grouped together, e.g.:

  \begin{verbatim}
    "slides" : ["slide1.png", "slide2.png", "slide3.png", "slide4.png"]
  \end{verbatim}

  The sprite object currently has 2 parameters - the internal name, and the
  file to load and so can look like:

  \begin{verbatim}
    {
      "name" : spriteName,
      "file" : spriteFilename
    }
  \end{verbatim}

  ``spriteName'' is the internal name of the sprite, and spriteFilename is 
  the filename to load from the VM.

\section{Coordinates}
  A brief aside on coordinates within the graphics application. Below,
  everything is given in non-normalised cartesian coordinates. By
  ``non-normalised'' I mean that an increment of one in the x or y dimensio
  means one pixel in that direction. What's more the origin is the centre
  of the window. This means that non-normalised coordinates will range from
  $-w/2 < x < w/2$, $-h/2 < y < h/2$.

  To allow for easy scaling to fullscreen (as any good screensaver should)
  the graphics app also supports normalised coordinates. These are fractions
  of the ``drawable width'' and ``drawable height'', and mean that the
  coordinates are in the range $-0.5 < x < 0.5$, $-0.5 < y < 0.5$. The 
  concept of the ``drawable'' sizes is to do with keeping a fixed aspect 
  ratio, regardless of the target platforms resolution. Currently this 
  aspect ratio is fixed at 16:10, so the coordinates respect these sizes.


\section{Objects} 
\subsection{Overview}
  ``objects'' is a list of json objects (or views, but we first discuss the
  simplistic case of one view), which each describe a control/widget
  or an 'object'. This json object is only required to specify a type of 
  object with key ``type'' and a value being one of the names described in 
  the sections below. The ``objects'' entry would therefore look akin to 
  this:

  \begin{verbatim}
    "objects" :
    [
      {
        "type" : "coolObject",
        //"coolObject" parameters
      },
      {
        "type" : "awesomeObject",
        //"awesomeObject" parameters
      }
    ]
  \end{verbatim}

  The objects currently supported by the graphics app are:
  \begin{enumerate}
    \item boincValue
    \item stringDisplay
    \item spriteDisplay
    \item slideshow
    \item gridshow
    \item panSprite
  \end{enumerate}

  To use them simply set the ``type'' parameter to the correspending name.
  The positioning of objects is always given in pixels and is cartesian
  (bottom-left origin).
  They are described soon, but first we must discuss ``views''.

  \subsection{Views}
  Simply put a view is a collection of objects that you want to be
  displayed at any given time. To switch between views you press any of the
  number keys (NOTE - only tested on Ubuntu). To set up a view in the config
  file, you simply make the top level node ``objects'' a list of lists of
  objects. Each list is then a view, and they are displayed in the order
  that they are declared (with the first being the default). This would look
  something like this:

  \begin{verbatim}
    objects:
    [
      [
        //View 1
        {
          //Object 1 in view 1
        },
        {
          //Object 2 in view 1
        }
      ],
      [
        //View 2
        {
          //Object 1 in view 2
        },
        {
          //Object 2 in view 2
        }
      ]
      //etc
    ]
  \end{verbatim}

  If you do not want to have multiple views, then just set the objects up
  as given in the ``objects'' section.

  \subsection{The Objects}

  \subsubsection{boincValue}
  This object displays a boinc parameter at a given place on the screen.
  The current supported Boinc parameters are:

  \begin{enumerate}
    \item username
    \item credit
  \end{enumerate}

  Parameters required for this object are the prefix (text prefixing the
  value), the dimensions and the value type. The value types accepted are
  given EXACTLY above. An example of the json object would be:

  \begin{verbatim}
    {
      "type"       : "boincValue",
      "dimensions" : {"x" : 40, "y" : 50},
      "valueType"  : "username", //could be "credit"
      "prefix"     : "The current user is: "
    }
  \end{verbatim}

  \subsubsection{stringDisplay}
  This object displays a collection of key-value pairs, split apart by
  a delimiter. The position given by the dimensions describes the bottom
  left of the top string. The strings are given as a map of key value pairs.
  The keys should be strings, but the values can be strings or numbers. Any
  number will be interpretted as an energy in GeV and printed as such.
  This map can either be given internally or externally. 

  To do it internally provide the parameter ``strings'', e.g.:

  \begin{verbatim}
    {
      "type"       : "stringDisplay",
      "dimensions" : {"x" : 400, "y" : 10},
      "strings"    :
      {
        "Energy"  : 12,
        "Cuts"    : "None"
        "Process" : "ee->mu mu"
      },
      "delimiter"  : ": "
    }
  \end{verbatim}
  
  To do it externally provide the boolean ``external'' as true, the 
  parameter ``resource'' as the internal name of the resource being used 
  and ``node'' as the to use from within that resource, e.g.:
  
  \begin{verbatim}
    {
      "type"       : "stringDisplay",
      "dimensions" : {"x" : 400, "y" : 10},
      "delimiter"  : ": ",
      "external"   : true,
      "resource"   : resourceName,
      "node"       : nodeInResource
    }
  \end{verbatim}

  If you wish to take a block of these strings but only display (for
  example) 3 at a time, then you can (optionally) set the parameter
  ``maxLines''. If you do do this, though, as the program isn't that smart 
  you must provide ``lineWidth'' as well, i.e. the spacing (in characters) 
  of the longest line. For example this could be:

  \begin{verbatim}
    {
      "type"       : "stringDisplay",
      "dimensions" : {"x" : 400, "y" : 10},
      "strings"    :
      {
        "Energy"  : 12,
        "Cuts"    : "None"
        "Process" : "ee->mu mu"
      },
      "delimiter"  : ": ",
      "maxLines"   : 1,
      "lineWidth"  : 18
    }
  \end{verbatim}


  \subsubsection{spriteDisplay}
  This object displays a single sprite at a given position. The required
  parameters are: the internal name of the sprite and the dimensions of the
  display (scaling is automatically supported). For example this could be:

  \begin{verbatim}
    {
      "type"       : "spriteDisplay",
      "sprite"     : "mySprite",
      "dimensions" : {"x" : 0, "y": 0, "w" : 100, "h" : 100}
    }
  \end{verbatim}

  \subsubsection{slideshow}
  This object displays a slideshow of sprites at a given position. The 
  required parameters are: the internal name of the group, the dimensions
  of the display (scaling is automatically supported) and the time between
  slides (given as a number of frames). The dimensions can either be the
  string ``fullscreen'', or a normal dimensions object. For example the
  slideshow object could be:

  \begin{verbatim}
    {
      "type"       : "slideshow",
      "sprites"    : "mySlides",
      "dimensions" : {"x" : 0, "y": 0, "w" : 100, "h" : 100},
      "timeout"    : 100
    }
  \end{verbatim}

  The object with the key of ``dimensions'' could simply be the string
  ``fullscreen'' if the slideshow was wanted fulscreen.

  \subsubsection{gridshow}
  This objects displays a slideshow of a grid of sprites. That is, if you
  give it a group of 16 pictures, telling it to display a 2x2 grid, it will
  cycle through the 16, four at a time. The setup is very similar to a
  slideshow object, in fact if you set up a 1x1 gridshow, it will behave
  identically to a slideshow.
  The required parameters are: the internal name of the group, the position,
  the individual cell dimensions (given as normal "w" and "h"), the number
  of cells wide that the grid will be, and the total number of cells to 
  display per slide. For example this could be:

  \begin{verbatim}
    {
      "type"       : "gridshow",
      "dimensions" : { "x": 627, "y": 0, "w": 299, "h": 270, 
                       "cellsWide" : 1 },
      "sprites"    : "histograms",
      "timeout"    : 100,
      "numCells"   : 2
    }  
  \end{verbatim}

  \subsubsection{panSprite}
  This object displays a single sprite by drawing an area of the sprite and
  panning across to display all of it. It can pan either vertically or
  horizontally depending on the settings provided. It accepts both
  normalised and non-normalised coordinates. It requires the drawn
  dimensions, the period (in frames) of panning and a value for the width
  or height that is grabbed from the image (we call this the display width).
  If both a display height and display width are provided there will be an
  error and the app will exit. The non-specified display dimension is taken
  as the full length of that axis. For example, for a vertically large image
  to pan vertically you could write:


  \begin{verbatim}
    {
      "type"       : "panSprite",
      "dimensions" : { "x": 627, "y": 0, "w": 300, "h": 270, 
                       "displayH" : "270" },
      "sprite"    : "tallImage",
      "period"    : 100
    }  
  \end{verbatim}


\end{document}
