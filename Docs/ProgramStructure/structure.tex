\documentclass[a4paper]{article}

\newcommand{\cernvmgraphics}{CernVM-Graphics}
\newcommand{\boinc}{BOINC}

\title{\cernvmgraphics{} Structure}
\author{Ben Page}

\begin{document}
\maketitle

\section{Entry Point and Initialisation}
  \cernvmgraphics{} begins, as with all C++ programs, in the main function.
  In main.cpp this is at the bottom of the file. Unlike a few functions this
  is fairly short and just sets up the boinc "hooks" as well as parses
  commandline arguments. ``--config='' forces a configuration file as
  specified on the command line, but this may not work too well now
  everything is grabbed from the VM. 
  
  ``boinc\_init\_graphics\_diagnostics'' calls the function 
  ``app\_graphics\_init'' and ``boinc\_graphics\_loop'' enters rendering 
  loop, ``app\_graphics\_render''. These are \boinc{} specified names.

  ``app\_graphics\_init'' does some basic initialisation, setting up the
  file downloader, loading fonts, setting the pause variable to false and
  sets up the error/debugging views. Essentially setting up most everything
  in the global scope that needs specific information. This is also where
  you find the message ``Waiting for VM''.

\section{Rendering Loop and Updating}
  The rendering loop itself is fairly verbose and simply calls a lot of
  other functions, relying on the current ``view'' being already populated.
  At the top it does some basic management - only updating the time if we
  aren't paused (all working code only gets a ``reported time'' rather than
  the actual time), and it also updates the shared memory. The next section
  downloads a new index.json every ``updatePeriod'' seconds and hooks a
  success function ``updateConfiguration''. After this we find the rendering
  code which simply runs through the objects in the active view. 

  The key function for updating is ``updateConfiguration'', found near the
  top of the source. It is an example of a FileDownloader ``responseFunc''
  (see networking.h). In principle this is fairly simple, it is called
  after downloading a configuration file and depending on if anything has
  changed it will ``update'' the configuration. This means loading new
  ``sprites'', ``objects'' and ``resources''. If the configuration doesn't
  change then we check for changes in resources. If the resources have
  changed then we download them all again.




\end{document}
